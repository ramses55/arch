\documentclass[a4paper]{article}
\usepackage[utf8]{inputenc}
%\usepackage{lmodern}
\usepackage{amsmath}
\usepackage[russian]{babel}
\usepackage{graphicx}
\usepackage{color}
\usepackage{subfiles}
\usepackage{float}




\usepackage[a4paper, total={6.3in,8in}]{geometry}


\title{Распознавание образов}

\date{}

\begin{document}

\maketitle

\section{Общие слова}

На изображении есть линейка, пакет с надписями и сама кость или несколько её фрагментов. Надо определить обуглена ли она, загрязнена ли, её размер. Считать надпись с пакета (qr код в том числе). Надо построить целый pipeline что, человек делает фото и загружает его, а система уже всё обрабатывает. Надо будет использовать OCR, YandexCloud.

\section{Идея решения}

Сейчас предлагается найти прямоугольную область внутри двух линеек (на одной картинке есть всего одна линейка, пока предлагается на неё забить).  В этой области будут кости и пакет. Пакет вроде успешно находится. Тогда остаётся область пакета залить белым цветом, а фрагменты костей просто выделить по цвету (буквально брать не белый цвет). Из полученной маски уже будет относительно легко получить все области. Эти области можно уже классифицировать на обугленный/ не обугленный, кость/зуб/ногть и так далее.

После надо узнать масштаб, чтобы определить размер фрагмента. Масштаб можно узнать по линейке или этикетке. Надо попробовать оба способа, хотя сейчас мне симпатичнее вариант с линейкой, так как на некоторых картинках пакет явно мятый и размер этикетки искажен.

\section{Последовательность этапов}

Здесь записываются этапы решения задачи, выполненные этапы отмечаются.



\begin{enumerate}
	\item Найти на изображениях пакет коричневого цвета.
	\item Найти линейку --- написать функцию для экспериментов в utils
	\item Нахождение масштаба (сколько мм один пиксель)
\end{enumerate}



\section{Этапы}


\subsection{Нахождение пакета}

Сейчас надо попытаться найти пакет, ориентируясь на его цвет или форму. Есть проблема, что цвет пакета разнится от картинки к картинке, и на некоторых картинках есть области (например, участки стола) с примерно таким же цветом. Поэтому предлагается брать контур с максимальной площадью. Пока это вроде работает, хотя и в некоторых случаях часть кости попадает, потому что она находится в непосредственной близости к пакету. Результат --- прямоугольник с наименьшей площадью, который вмещает в себя выделенную область с максимальной площадью.


То есть сейчас алгоритм такой:


\begin{enumerate}
	\item Выделение маски по цвету с помощью cv2.inRange() (после блюр 21$\times$21, алгоритм чувствителен к этому параметру)
	\item Нахождение конутров маски
	\item Выбор контура с наибольшей площадью
	\item Подгон прямоугольника с наименьшей площадью под эту область
	\item Специальная функция вырезает получившийся прямоугольник из картинки (функция нужна, потому что прямоугольник может быть наклонен)
\end{enumerate}



\subsection{Нахождение масштаба}
На каждой картинке есть линейка криминалистическая. На ней есть черные прямоугольники, длина которых равна одному сантиметру. Проблема в том, что они все немного разных размеров. Сейчас есть идея просто взять среднее значение. Если такая точность будет приемлема, то остановимся на этом.

\end{document}
